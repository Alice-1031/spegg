\documentclass[11pt]{article}
\pdfoutput=1
\usepackage{graphicx}
\usepackage[authoryear,round]{natbib}
\usepackage{caption}
\usepackage{subcaption}
\usepackage[mathlines]{lineno}
\usepackage{setspace}
\usepackage{mathrsfs}
\usepackage{rotating}
\usepackage{url}
\usepackage{amssymb}
\usepackage{multirow}
\usepackage[none]{hyphenat}
\usepackage{algpseudocode}
\DeclareGraphicsExtensions{.pdf,.png,.jpg}
\setlength{\topmargin}{-.5in}
\setlength{\textheight}{9in}
\setlength{\oddsidemargin}{.125in}
\setlength{\textwidth}{6.25in}
\doublespacing
\pagestyle{empty}

\begin{document}
\begin{algorithmic}
\Large{}
\For{$1 \leq S\leq $Number of Replicates}
\For{$1 \leq i\leq $Number of Individuals} 
	\State $parameter \gets $ Parameter values for 
	\State \hspace{1.75in} simulation $S$
	\State trait value of individual $i \gets foo(parameter)$
\EndFor
\EndFor
\end{algorithmic}
\begin{center}
\Large{(A)}
\end{center}

\begin{algorithmic}
\Large{}
\For{$1 \leq i\leq $ Number of individuals} 
 in\\
\hspace{1.5in} parallel
\Large{}
	\State $D \gets $ Replicate (patch) of individual $i$
	\State $parameter \gets $ Parameter values from
	\State \hspace{1.75in} deme $D$
	\State trait value of individual $i \gets foo(parameter)$
\EndFor
\end{algorithmic}
\begin{center}
\Large{(B)}\\
\end{center}

\begin{singlespace}
\noindent Supplementary Figure S5. Pseudo-code illustrating the primary difference between individual-based models following (A) a serial implementation and (B) a parallelized implementation using sPEGG. Each segment of pseudo-code represents calculations performed in Figure 1D of the main text to update the phenotypes of individuals using the user-supplied function \texttt{foo()} for each time step. In (A), when there are $n$ individuals and $p$ patches, then $np$ operations must be performed sequentially (one after another). By contrast, in (B), if there are $k$ parallel cores, then $k$ operations can be performed simultaneously, and under ideal conditions, evaluating the loop takes $np/k$ steps. In effect, the operations in (B) allow us to ``flatten'' the nested loop described in (A), enabling fine-grained parallelism of the individual-based model.
\end{singlespace}
\end{document}

